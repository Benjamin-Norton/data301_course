\documentclass[xcolor=svgnames, 10pt, handout]{beamer}

\usepackage[utf8]{inputenc}
\usepackage[T1]{fontenc}
\usepackage[english]{babel}

%\usepackage{verbatim}

\usepackage[export]{adjustbox}

\usepackage{
    amsmath,
    amsfonts,
    etex,
    fancyvrb,
    graphicx,
    multicol,
    pifont,
    setspace,
    soul,
    spverbatim,
    textcomp,
    xcolor,
    xspace
}

\usepackage{tikz}
\usetikzlibrary{shadows}

%%%SETUP%%%
\hypersetup{
     colorlinks = true,
     linkcolor = blue,
     anchorcolor = blue,
     citecolor = blue,
     filecolor = blue,
     urlcolor = blue
     }
     
%%%THEOREMS%%%
\theoremstyle{example}
\newtheorem*{exercise}{Exercise}
\newtheorem*{question}{Question}
\newtheorem*{answer}{\emph{Answer}}
\newtheorem*{notation}{Notation}

%%%TWEAKS%%%
\setlength\arraycolsep{4pt}
\addtolength\fboxsep{10pt}
\setstretch{1.4}
\setbeamersize{description width=3em}
\setbeamersize{text margin left=.5cm,text margin right=.5cm} 
\renewcommand{\emph}{\alert}
\renewcommand{\arraystretch}{1.2}
\renewcommand{\tabcolsep}{4pt}
\setbeamercolor{alerted text}{fg=magenta}


\graphicspath{{img/}}

%for straight quotes in verbatim:
\usepackage{upquote,textcomp}

%turn off navigation symbols
\beamertemplatenavigationsymbolsempty
\setbeamertemplate{footline}[frame number]

%title page

\author
  [Dr.\ Irene Vrbik]
  {Dr.\ Irene Vrbik}

\date
  {}

\institute
  {University of British Columbia Okanagan \newline \texttt{irene.vrbik@ubc.ca}}
  
\definecolor{iyellow}{RGB}{255, 162, 23}
\definecolor{sgreen}{RGB}{118, 191, 138}

\newcommand{\yellow}[1]{\textcolor{iyellow}{#1}}
\newcommand{\red}[1]{\textcolor{red}{#1}}
\newcommand{\green}[1]{\textcolor{ForestGreen}{#1}}
\newcommand{\blue}[1]{{\textcolor{blue}{#1}}}
\newcommand{\orange}[1]{{\textcolor{orange}{#1}}}
\newcommand{\bblue}[1]{\textcolor{SteelBlue!90!gray}{#1}} % beamer blue
\newcommand{\purple}[1]{{\textcolor{purple}{#1}}}

\newcommand{\el}{\\[1em]\pause}
\newcommand{\nl}{\\[1em]}
\newcommand{\define}[1]{\textbf{\textcolor{orange}{#1}}}

%\newcommand{\answer}[1]{\textit{\textbf{\textcolor{iyellow}{#1}}}}

\newcommand{\command}[1]{\texttt{\textbf{\textcolor{DarkMagenta}{#1}}}}
\newcommand{\ipic}[2]{\includegraphics[width={#2}\textwidth]{#1}}
\newcommand{\cell}[1]{{\sf \textbf{\textcolor{DarkMagenta}{#1}}}}
\newcommand{\ra}{$\rightarrow$}

\newcommand{\ft}[1]{\frametitle{#1}}


\newenvironment{allintypewriter}{\ttfamily}{\par}
\newcommand{\bs}{$\backslash$}

\newcommand*\keystroke[1]{%
  \tikz[baseline=(key.base)]
    \node[%
      draw,
      fill=white,
      drop shadow={shadow xshift=0.25ex,shadow yshift=-0.25ex,fill=black,opacity=0.75},
      rectangle,
      rounded corners=2pt,
      inner sep=1pt,
      line width=0.5pt,
      font=\scriptsize\sffamily
    ](key) {#1\strut}
  ;
}

% timed answer
\newcommand{\tans}[2]{\textbf<#1>{\textit<#1>{{\color<#1>{iyellow}{#2}}}}}


\makeatletter
\g@addto@macro\normalsize{%
  \setlength\abovedisplayskip{0.4em}
  \setlength\belowdisplayskip{0.4em}
  \setlength\abovedisplayshortskip{0.2em}
  \setlength\belowdisplayshortskip{0.2em}
}
\makeatother


\newcommand{\cmark}{{\Large\color{green}\ding{51}}}%
\newcommand{\xmark}{{\Large\color{red}\ding{55}}}%

\newcommand{\pcmark}{\onslide<+->{\cmark}}
\newcommand{\pxmark}{\onslide<+->{\xmark}}

\newcommand{\by}{\overline{y}}
\newcommand{\ty}{\tilde{y}}

\title
  [R]
  {R part III}
  \subtitle{Data Analysis with R}


\begin{document}


\maketitle



\begin{frame}{Introduction}
\begin{itemize}
\item Now that we have seen some of the basic functionalities of R, we can now discuss some of the important statistical tests that are commonly used. 
\item The 
\end{itemize}

\end{frame}


%%%%%%%%%%%%%%%%%%%%%%%%%%%%%%%%%%%%%%%%%%%%%%%%%%%%%%%%%%%%%%%%%%%%%%%%%%%%%%%%%%%%%%%%%%%%%%%%%%%%
\begin{frame}[fragile]{Confidence Intervals}
Consider the following statement
\begin{quotation}
``62\% of US college students miss a class due to excessive drinking.  The result is accurate with 1.7 percentage points 19 times out of 20.''
\end{quotation}
\vfill
Unpacking this statement:
\begin{enumerate}
\item \emph{62} is the estimated percentage.
\item \emph{1.7} is the margin of error.
\item \emph{19 times out of 20} is the stated confidence and $100\% \frac{19}{20} = 95\%$.
\end{enumerate}
\vfill
We call this a \emph{95\% confidence interval}: $(60.3,\, 63.7)$
\vfill
\end{frame}
%%%%%%%%%%%%%%%%%%%%%%%%%%%%%%%%%%%%%%%%%%%%%%%%%%%%%%%%%%%%%%%%%%%%%%%%%%%%%%%%%%%%%%%%%%%%%%%%%%%%


%%%%%%%%%%%%%%%%%%%%%%%%%%%%%%%%%%%%%%%%%%%%%%%%%%%%%%%%%%%%%%%%%%%%%%%%%%%%%%%%%%%%%%%%%%%%%%%%%%%%
\begin{frame}[fragile]
\vfill
\begin{block}{General Form of Confidence Interval:}
\begin{align*}
\text{point estimate }&\pm \text{ margin of error}\\
(\hat\mu-m.e.&,\, \hat\mu+m.e.)
\end{align*}
\end{block}
\vfill
Interpretation of a 95\% confidence interval: we are 95\% confident that the interval will contain the true value of the parameter.
\vfill
\end{frame}
%%%%%%%%%%%%%%%%%%%%%%%%%%%%%%%%%%%%%%%%%%%%%%%%%%%%%%%%%%%%%%%%%%%%%%%%%%%%%%%%%%%%%%%%%%%%%%%%%%%%


%%%%%%%%%%%%%%%%%%%%%%%%%%%%%%%%%%%%%%%%%%%%%%%%%%%%%%%%%%%%%%%%%%%%%%%%%%%%%%%%%%%%%%%%%%%%%%%%%%%%
\begin{frame}[fragile]{Hypothesis Testing}
\emph{Hypothesis testing} is used to determine if a relationship exists between two sets of data and make decisions/conclusions about that relationship.
\vfill
\emph{Hypothesis testing is useful for}
\begin{enumerate}
\item \emph{business} in determining effectiveness of marketing, identifying customer buying properties, online advertising optimization.
\item \emph{science and social science} in determining if data sets match a model, understanding scientific process based on collected data values, analysis of study data.
\end{enumerate}
\vfill
\end{frame}
%%%%%%%%%%%%%%%%%%%%%%%%%%%%%%%%%%%%%%%%%%%%%%%%%%%%%%%%%%%%%%%%%%%%%%%%%%%%%%%%%%%%%%%%%%%%%%%%%%%%


%%%%%%%%%%%%%%%%%%%%%%%%%%%%%%%%%%%%%%%%%%%%%%%%%%%%%%%%%%%%%%%%%%%%%%%%%%%%%%%%%%%%%%%%%%%%%%%%%%%%
\begin{frame}[fragile]{Probability Distribution Functions}
\begin{description}
\item[\texttt rnorm] Random Normal variables.
\item[\texttt dnorm] Evaluate normal probability density/
\item[\texttt pnorm]  Evaluates normal distribution function 
\item[\texttt qnorm] Quantile function
\item[\texttt d] Density.
\item[\texttt r] Random number generation.
\item[\texttt p] Cumulative distribution.
\item[\texttt q] Quantile function.
\end{description}
Default distribution is ${\rm mean}=0$ and ${\rm sd}=1$.
\end{frame}
%%%%%%%%%%%%%%%%%%%%%%%%%%%%%%%%%%%%%%%%%%%%%%%%%%%%%%%%%%%%%%%%%%%%%%%%%%%%%%%%%%%%%%%%%%%%%%%%%%%%


%%%%%%%%%%%%%%%%%%%%%%%%%%%%%%%%%%%%%%%%%%%%%%%%%%%%%%%%%%%%%%%%%%%%%%%%%%%%%%%%%%%%%%%%%%%%%%%%%%%%
\begin{frame}[fragile]{Hypothesis Testing Steps}
\begin{enumerate}
\item Declare hypotheses statement and null hypothesis.
\item Decide on a test statistic.
\item Use P-value and/or confidence interval to make decision/conclusion.
\begin{enumerate}
\item A p-value of 0.05 ``signifies that if the null hypothesis is true, and all other assumptions made are valid, there is a 5\% chance of obtaining a result at least as extreme as the one observed.''
\end{enumerate}
\item Data is used as evidence.  Perform a test in order to make a decision:  reject the null hypothesis or fail to reject the null hypothesis.
\end{enumerate}
\vfill
Note: We cannot \emph{prove} that the null hypothesis is true or false. We can only show that there is evidence to suggest one conclusion or another. 
\end{frame}
%%%%%%%%%%%%%%%%%%%%%%%%%%%%%%%%%%%%%%%%%%%%%%%%%%%%%%%%%%%%%%%%%%%%%%%%%%%%%%%%%%%%%%%%%%%%%%%%%%%%

%%%%%%%%%%%%%%%%%%%%%%%%%%%%%%%%%%%%%%%%%%%%%%%%%%%%%%%%%%%%%%%%%%%%%%%%%%%%%%%%%%%%%%%%%%%%%%%%%%%%
\begin{frame}[fragile]{Assumptions}
\vfill
There are assumptions that need to be met before performing statistical tests.
\vfill
For the one sample:
\begin{enumerate}
\item Population of interest is normally distributed.
\item Independent random samples are taken.
\end{enumerate}
\vfill
For the two sample case:
\begin{enumerate}
\item The two samples are independent.
\item Populations of interest are normally distributed.
\end{enumerate}
\vfill
\end{frame}
%%%%%%%%%%%%%%%%%%%%%%%%%%%%%%%%%%%%%%%%%%%%%%%%%%%%%%%%%%%%%%%%%%%%%%%%%%%%%%%%%%%%%%%%%%%%%%%%%%%%


%%%%%%%%%%%%%%%%%%%%%%%%%%%%%%%%%%%%%%%%%%%%%%%%%%%%%%%%%%%%%%%%%%%%%%%%%%%%%%%%%%%%%%%%%%%%%%%%%%%%
\begin{frame}[fragile]{One Sample Test}
\vfill
A \emph{one sample test} is used when a sample is compared to a model or known population/estimate.
\vfill
As an example, using the car data test if the average mileage is different than 10km/L.
\vfill
\end{frame}
%%%%%%%%%%%%%%%%%%%%%%%%%%%%%%%%%%%%%%%%%%%%%%%%%%%%%%%%%%%%%%%%%%%%%%%%%%%%%%%%%%%%%%%%%%%%%%%%%%%%


%%%%%%%%%%%%%%%%%%%%%%%%%%%%%%%%%%%%%%%%%%%%%%%%%%%%%%%%%%%%%%%%%%%%%%%%%%%%%%%%%%%%%%%%%%%%%%%%%%%%
\begin{frame}[fragile]{One Sample Test: Hypotheses Statements}
Null hypothesis ($H_0$) always contains a statement of no change ($=$).
\vfill
Alternatively hypothesis ($H_A$) can be one sided ($<$ or $>$) or two sided ($\neq$).
\begin{enumerate}
\item $H_0$: $\mu = \verb|test_number|$,
\item $H_A$: $\mu \neq \verb|test_number|$.
\end{enumerate}
\vfill
Car mileage example
\begin{enumerate}
\item $H_0$: $\mu = \verb|10|$,
\item $H_A$: $\mu \neq \verb|10|$.
\end{enumerate}
More generically, we use $H_0: \mu = \mu_0$ where $\mu_0$ is the hypothesized mean.
\vfill
\end{frame}
%%%%%%%%%%%%%%%%%%%%%%%%%%%%%%%%%%%%%%%%%%%%%%%%%%%%%%%%%%%%%%%%%%%%%%%%%%%%%%%%%%%%%%%%%%%%%%%%%%%%


%%%%%%%%%%%%%%%%%%%%%%%%%%%%%%%%%%%%%%%%%%%%%%%%%%%%%%%%%%%%%%%%%%%%%%%%%%%%%%%%%%%%%%%%%%%%%%%%%%%%
\begin{frame}[fragile]{One Sample Test: Calculate Test Statistic}
\vfill
For the one sample test the $t$-test statistic is calculated as:
$$t = \dfrac{\bar y - \mu_0}{\dfrac{s}{\sqrt n}}$$
\vfill
$\overline y$ is a sample mean, $s$ is sample standard deviation, $n$ is sample size, $\mu_0$ is hypothesized mean value
\vfill
\begin{Verbatim}[xleftmargin=2em, xrightmargin=1.5em, frame=single, label=R code, framesep=0.5em]
car_data <- read.csv("car_data.csv")
t.test(x=car_data$km.L,
       alternative=c("two.sided"), mu=10)
\end{Verbatim}
\end{frame}
%%%%%%%%%%%%%%%%%%%%%%%%%%%%%%%%%%%%%%%%%%%%%%%%%%%%%%%%%%%%%%%%%%%%%%%%%%%%%%%%%%%%%%%%%%%%%%%%%%%%


%%%%%%%%%%%%%%%%%%%%%%%%%%%%%%%%%%%%%%%%%%%%%%%%%%%%%%%%%%%%%%%%%%%%%%%%%%%%%%%%%%%%%%%%%%%%%%%%%%%%
\begin{frame}[fragile]{One Sample Test: Decision and Conclusion (using $P$-value)}
\vfill
If $p{\rm-value} > 0.05$, the probability of seeing a sample mean more extreme is not that unlikely.
\begin{enumerate}
\item Fail to reject the null hypothesis.
\item There is insufficient  evidence to suggest that the mean value is less than/greater than/different than the test value (will depend on alternative hypothesis)
\end{enumerate}
\vfill
If $p{\rm-value} < 0.05$,
\begin{enumerate}
\item Reject the null hypothesis.
\item There is statistically significant evidence to suggest that the mean value is less than that/greater than/ different than the test value  (will depend on alternative hypothesis).
\end{enumerate}
\vfill
\end{frame}
%%%%%%%%%%%%%%%%%%%%%%%%%%%%%%%%%%%%%%%%%%%%%%%%%%%%%%%%%%%%%%%%%%%%%%%%%%%%%%%%%%%%%%%%%%%%%%%%%%%%


%%%%%%%%%%%%%%%%%%%%%%%%%%%%%%%%%%%%%%%%%%%%%%%%%%%%%%%%%%%%%%%%%%%%%%%%%%%%%%%%%%%%%%%%%%%%%%%%%%%%
\begin{frame}[fragile]{One Sample Test: Decision and Conclusions (using $P$-value)}
\begin{Verbatim}[xleftmargin=2em, xrightmargin=1.5em, frame=single, label=R code, framesep=0.5em, fontsize=\small, commandchars=\\\{\}]
> t.test(x=car_data$km.L, alternative=c("two.sided"),mu=10)
            
            One Sample t-test
            
data: car_data$km.L
t = 1.608, df = 29, \emph{p-value = 0.1187}
alternative hypothesis: true mean is not equal to 10
95 percent confidence interval:
    9.90338  10.80729
sample estimates:
mean of x
  10.35533
\end{Verbatim}
\end{frame}
%%%%%%%%%%%%%%%%%%%%%%%%%%%%%%%%%%%%%%%%%%%%%%%%%%%%%%%%%%%%%%%%%%%%%%%%%%%%%%%%%%%%%%%%%%%%%%%%%%%%


%%%%%%%%%%%%%%%%%%%%%%%%%%%%%%%%%%%%%%%%%%%%%%%%%%%%%%%%%%%%%%%%%%%%%%%%%%%%%%%%%%%%%%%%%%%%%%%%%%%%
\begin{frame}[fragile]{One Sample Test: Decision and Conclusions (using $P$-value)}
\vfill
$P$-value $=0.1187 > 0.05 \implies \text{ fail to reject the null hypothesis}$.
\vfill
There is insufficient evidence to suggest that the mean mileage differs from 10 km/L.
\vfill
Note: Unable to claim that either the null or alternative hypothesis is true.  Can only reject or fail to reject the null hypothesis.
\vfill
\end{frame}
%%%%%%%%%%%%%%%%%%%%%%%%%%%%%%%%%%%%%%%%%%%%%%%%%%%%%%%%%%%%%%%%%%%%%%%%%%%%%%%%%%%%%%%%%%%%%%%%%%%%


%%%%%%%%%%%%%%%%%%%%%%%%%%%%%%%%%%%%%%%%%%%%%%%%%%%%%%%%%%%%%%%%%%%%%%%%%%%%%%%%%%%%%%%%%%%%%%%%%%%%
\begin{frame}[fragile]{One Sample Test: Decision and Conclusions (using CI)}\small
\begin{Verbatim}[xleftmargin=2em, xrightmargin=1.5em, frame=single, label=R code, framesep=0.5em, fontsize=\footnotesize, commandchars=\\\{\}]
> t.test(x=car_data$km.L, alternative=c("two.sided"),mu=10)
            One Sample t-test
data: car_data$km.L
t = 1.608, df = 29, p-value = 0.1187
alternative hypothesis: true mean is not equal to 10
\emph{95 percent confidence interval:}
\emph{    9.90338  10.80729}
sample estimates:
mean of x
  10.35533
\end{Verbatim}
Can also make a conclusion (reject or fail to reject) based on the confidence interval. We are 95\% confident that the true mean mileage of the car lies within those bounds. 
\vfill
Since 10 km/L is within those bounds, fail to reject the null hypothesis.
\vfill
\end{frame}
%%%%%%%%%%%%%%%%%%%%%%%%%%%%%%%%%%%%%%%%%%%%%%%%%%%%%%%%%%%%%%%%%%%%%%%%%%%%%%%%%%%%%%%%%%%%%%%%%%%%


%%%%%%%%%%%%%%%%%%%%%%%%%%%%%%%%%%%%%%%%%%%%%%%%%%%%%%%%%%%%%%%%%%%%%%%%%%%%%%%%%%%%%%%%%%%%%%%%%%%%
\begin{frame}[fragile]{Two Sample Unpaired}
\vfill
An unpaired (independent) \emph{two sample test} compares two independent samples to determine if there is a difference between the groups.
\vfill
\begin{example}
\begin{enumerate}
\item Compare effectiveness of two different drugs tested on two sets of patients.\\
\item Experiment versus control samples.
\end{enumerate}
\end{example}
\vfill
\end{frame}
%%%%%%%%%%%%%%%%%%%%%%%%%%%%%%%%%%%%%%%%%%%%%%%%%%%%%%%%%%%%%%%%%%%%%%%%%%%%%%%%%%%%%%%%%%%%%%%%%%%%


%%%%%%%%%%%%%%%%%%%%%%%%%%%%%%%%%%%%%%%%%%%%%%%%%%%%%%%%%%%%%%%%%%%%%%%%%%%%%%%%%%%%%%%%%%%%%%%%%%%%
\begin{frame}[fragile]{Two Sample Unpaired Example
Hypothesis Statement}
Using the \verb|beaver2| dataset in R, test the hypothesis that there is no difference between the mean active temperature and the mean non-active temperatures.
\begin{align*}
H_0 : \mu_1 = \mu_2 \to \mu_1 -\mu_2 = 0 \\
H_A : \mu1 \neq \mu_2 \to \mu_1 - \mu_2 \neq 0
\end{align*}
\end{frame}
%%%%%%%%%%%%%%%%%%%%%%%%%%%%%%%%%%%%%%%%%%%%%%%%%%%%%%%%%%%%%%%%%%%%%%%%%%%%%%%%%%%%%%%%%%%%%%%%%%%%


%%%%%%%%%%%%%%%%%%%%%%%%%%%%%%%%%%%%%%%%%%%%%%%%%%%%%%%%%%%%%%%%%%%%%%%%%%%%%%%%%%%%%%%%%%%%%%%%%%%%
\begin{frame}[fragile]{Two Sample Unpaired Example Test Statistic}
\vfill
Use $t$-test statistics.
\vfill
\begin{Verbatim}[xleftmargin=2em, xrightmargin=1.5em, frame=single, numbers=left, label=Using P-value, framesep=0.5em, commandchars=\\\{\}]
# Need to set active to be a factor first.
beaver2$activ = as.factor(beaver2$activ)
# Perform unpaired test
t.test(temp{~}activ, data=beaver2,
       alternative=c("two.sided"), mu=0,
       paired=FALSE)
\end{Verbatim}
\vfill
\end{frame}
%%%%%%%%%%%%%%%%%%%%%%%%%%%%%%%%%%%%%%%%%%%%%%%%%%%%%%%%%%%%%%%%%%%%%%%%%%%%%%%%%%%%%%%%%%%%%%%%%%%%


%%%%%%%%%%%%%%%%%%%%%%%%%%%%%%%%%%%%%%%%%%%%%%%%%%%%%%%%%%%%%%%%%%%%%%%%%%%%%%%%%%%%%%%%%%%%%%%%%%%%
\begin{frame}[fragile]{Two Sample Unpaired Example Decision and Conclusions}
\begin{Verbatim}[xleftmargin=2em, xrightmargin=1.5em, frame=single, label=Using CI-value, framesep=0.5em, commandchars=\\\{\}, fontsize=\footnotesize]
> t.text(temp-activ, data=beaver2, alternative=c("two.sided"), mu=0,
paired=FALSE)
        welch Two Sample t-test
data:    temp by activ
t = -18.548,  df = 80.852, \emph{p-value < 2.2e-16}
alternative hypothesis: true difference in means is not equal to 0
95 percent condifence interval:
  -0.8927106  -0.7197342
sample estimates:
mean in group 0 mean in group 1
        37.09684        37.90306
\end{Verbatim}
The p-value $<<$ 0.05.
\vfill
Reject the null hypothesis. There is vert strong evidence to suggest that there is a difference between active and non-active temperatures. 
\end{frame}
%%%%%%%%%%%%%%%%%%%%%%%%%%%%%%%%%%%%%%%%%%%%%%%%%%%%%%%%%%%%%%%%%%%%%%%%%%%%%%%%%%%%%%%%%%%%%%%%%%%%

\begin{frame}
The two sample case tests a DIFFERENCE between the groups $(\mu_1 - \mu_2 \neq 0)$. \\[0.2in]
The CI stated on the previous slide is the CI for the difference, $\mu_1 - \mu_2$\\[0.2in]
We reject the null hypothesis because 0 is not contained in the interval. \\[0.2in]
If 0 was contained we would fail to reject the null hypothesis.
%
\end{frame}

%%%%%%%%%%%%%%%%%%%%%%%%%%%%%%%%%%%%%%%%%%%%%%%%%%%%%%%%%%%%%%%%%%%%%%%%%%%%%%%%%%%%%%%%%%%%%%%%%%%%
\begin{frame}[fragile]{Two Sample Paired Test}
A \emph{paired (dependent) two sample test} compares two dependent samples to see if there is a difference between the groups.
\begin{enumerate}
\item This test typically uses multiple measurements on one subject.
\item Also called a "repeated measures" test.
\end{enumerate}
\vfill
\begin{examples}
\begin{enumerate}
\item Affect of treatment on a patient (before and after)
\item Apply something to test subjects to see if there is an effect
\item Car example: Do cars get better mileage with different grades of gasoline?
\end{enumerate}
\end{examples}

\end{frame}
%%%%%%%%%%%%%%%%%%%%%%%%%%%%%%%%%%%%%%%%%%%%%%%%%%%%%%%%%%%%%%%%%%%%%%%%%%%%%%%%%%%%%%%%%%%%%%%%%%%%


%%%%%%%%%%%%%%%%%%%%%%%%%%%%%%%%%%%%%%%%%%%%%%%%%%%%%%%%%%%%%%%%%%%%%%%%%%%%%%%%%%%%%%%%%%%%%%%%%%%%
\begin{frame}[fragile]{Two Sample Paired Test Example Hypothesis Statement}
\vfill
The \verb|ahtlete.csv| dataset contains data on ten athletes and their speeds for the 100m dash before training (Training = 0) and after (Training = 1).
\vfill
Test the hypothesis that the training has no effect on the times of the athletes.  Test to see if the mean of the difference is different than 0.
\begin{align*}
H_0 &: d = 0\\
H_A &: d \neq 0
\end{align*}
\vfill
\end{frame}
%%%%%%%%%%%%%%%%%%%%%%%%%%%%%%%%%%%%%%%%%%%%%%%%%%%%%%%%%%%%%%%%%%%%%%%%%%%%%%%%%%%%%%%%%%%%%%%%%%%%


%%%%%%%%%%%%%%%%%%%%%%%%%%%%%%%%%%%%%%%%%%%%%%%%%%%%%%%%%%%%%%%%%%%%%%%%%%%%%%%%%%%%%%%%%%%%%%%%%%%%
\begin{frame}[fragile]{Two Sample Paired Test Example Test Statistic -- R Code}
\begin{Verbatim}[xleftmargin=2em, xrightmargin=1.5em, frame=single, label=Using CI-value, framesep=0.5em, commandchars=\\\{\}, fontsize=\small]
# Read the data
athlete = read.csv("athlete.csv", header=TRUE)

# Perform paired test
t.test(Time~Training, data=athlete, 
  alternative=c("two.sided"), my=0, paired=TRUE)
\end{Verbatim}
\end{frame}
%%%%%%%%%%%%%%%%%%%%%%%%%%%%%%%%%%%%%%%%%%%%%%%%%%%%%%%%%%%%%%%%%%%%%%%%%%%%%%%%%%%%%%%%%%%%%%%%%%%%


%%%%%%%%%%%%%%%%%%%%%%%%%%%%%%%%%%%%%%%%%%%%%%%%%%%%%%%%%%%%%%%%%%%%%%%%%%%%%%%%%%%%%%%%%%%%%%%%%%%%
\begin{frame}[fragile]{Two Sample Paired Test Example  Decision and Conclusion}
\begin{Verbatim}[xleftmargin=2em, xrightmargin=1.5em, frame=single, label=Using P-value, framesep=0.5em, commandchars=\\\{\}, fontsize=\footnotesize]
> t.test(Time~Training, data=athlete, alternative=c("two.sided"),
  mu=0, paired=TRUE)
        Paired t-test
data:    Time by Training
t = -0.12031, df = 9, \emph{p-value = 0.9069}
alternative hypothesis: true difference in means is not equal to 0
95 percent confidence interval:
-0.5544647  0.4984647
sample estimates:
mean of the differences
                 -0.028
\end{Verbatim}
The $p{\rm -value}>>0.05$.
\vfill
Fail to reject the null hypothesis.  There is insufficient evidence to suggest that there is a difference between pre and post training times.
\end{frame}
%%%%%%%%%%%%%%%%%%%%%%%%%%%%%%%%%%%%%%%%%%%%%%%%%%%%%%%%%%%%%%%%%%%%%%%%%%%%%%%%%%%%%%%%%%%%%%%%%%%%


%%%%%%%%%%%%%%%%%%%%%%%%%%%%%%%%%%%%%%%%%%%%%%%%%%%%%%%%%%%%%%%%%%%%%%%%%%%%%%%%%%%%%%%%%%%%%%%%%%%%
\begin{frame}[fragile]{Two Sample Paired Test Example Decision and Conclusion}
\begin{Verbatim}[xleftmargin=2em, xrightmargin=1.5em, frame=single, label=Using CI, framesep=0.5em, commandchars=\\\{\}, fontsize=\footnotesize]
> t.test(Time-Training, data=athlete, alternative=c("two.sided"), 
  mu=0, paired=TRUE)
        Paired t-test
data:    Time by training
t = -0.12031,  df = 9,  p-values = 0.9069
alternative hypothesis true difference in means is not equal to 0
-0.5544647   0.4984647
sample estimates:
mean of the differences
                 -0.028
\end{Verbatim}
The two sample case tests for a difference between the groups ($d \neq 0$).  The CI is for the difference.
\vfill
Fail to reject the null hypothesis because 0 is contained in the confidence interval.
\end{frame}
%%%%%%%%%%%%%%%%%%%%%%%%%%%%%%%%%%%%%%%%%%%%%%%%%%%%%%%%%%%%%%%%%%%%%%%%%%%%%%%%%%%%%%%%%%%%%%%%%%%%


%%%%%%%%%%%%%%%%%%%%%%%%%%%%%%%%%%%%%%%%%%%%%%%%%%%%%%%%%%%%%%%%%%%%%%%%%%%%%%%%%%%%%%%%%%%%%%%%%%%%
\begin{frame}[fragile]
\begin{question}
Which (if any) of the following are true?
\begin{enumerate}
\item Paired and unpaired $t$-tests are the same thing. \onslide<+-> \pxmark
\item Confidence intervals can be of any level of confidence (not just 95\%). \pcmark
\item Confidence intervals can be used to make a conclusion about a hypothesis test. \pcmark
\item Confidence intervals can be used to prove that the null hypothesis is false. \pxmark
\end{enumerate}
\end{question}
\end{frame}
%%%%%%%%%%%%%%%%%%%%%%%%%%%%%%%%%%%%%%%%%%%%%%%%%%%%%%%%%%%%%%%%%%%%%%%%%%%%%%%%%%%%%%%%%%%%%%%%%%%%


%%%%%%%%%%%%%%%%%%%%%%%%%%%%%%%%%%%%%%%%%%%%%%%%%%%%%%%%%%%%%%%%%%%%%%%%%%%%%%%%%%%%%%%%%%%%%%%%%%%%
\begin{frame}[fragile]
\begin{question}
Which (if any) of the following are true?
\begin{enumerate}
\item Unpaired t-tests test the difference between two means $\mu_1$ and $\mu_2$. 
\item Paired t-tests can be used to compare the difference between two measurements on the same subject.\pcmark
\item In both the paired and unpaired two sample cases, a confidence interval containing 0 would result in a decision of: fail to reject the null hypothesis.\pcmark (assuming $H_0: \mu_1 - \mu_2 = 0$, we could however have $H_0: \mu_1 - \mu_2 = \mu_d$ with $\mu_d\neq 0$
\item  In the one sample $t$-test, a confidence interval containing 0 would result in a decision of: fail to reject the null hypothesis.\pxmark
\end{enumerate}
\end{question}
\end{frame}
%%%%%%%%%%%%%%%%%%%%%%%%%%%%%%%%%%%%%%%%%%%%%%%%%%%%%%%%%%%%%%%%%%%%%%%%%%%%%%%%%%%%%%%%%%%%%%%%%%%%


%%%%%%%%%%%%%%%%%%%%%%%%%%%%%%%%%%%%%%%%%%%%%%%%%%%%%%%%%%%%%%%%%%%%%%%%%%%%%%%%%%%%%%%%%%%%%%%%%%%%
\begin{frame}[fragile]\small
\begin{question}
Which is the most appropriate test for each of the following?
\begin{enumerate}
\item Is the average student mark in courses 70\%?  \textit{one sample t-test}
\item Does a student's mark improve after studying?  \textit{two sample paired (same student)}
\item Has the average student height increased since 1990?  \textit{two sample unpaired (two distinct student populations)}
\item Does radiation reduce the size of tumours when used to treat patients?  \textit{two sample paired (same patient), although could argue against control/experiment groups in which case two sample unpaired}
\item Are college graduates better than high school graduates at standardized tests?   \textit{two sample unpaired }
%\item Is aspirin more effective than Tylenol for treating headaches?  \textit{ two sample unpaired }

\end{enumerate}
\end{question}
\end{frame}
%%%%%%%%%%%%%%%%%%%%%%%%%%%%%%%%%%%%%%%%%%%%%%%%%%%%%%%%%%%%%%%%%%%%%%%%%%%%%%%%%%%%%%%%%%%%%%%%%%%%




%%%%%%%%%%%%%%%%%%%%%%%%%%%%%%%%%%%%%%%%%%%%%%%%%%%%%%%%%%%%%%%%%%%%%%%%%%%%%%%%%%%%%%%%%%%%%%%%%%%%
\begin{frame}[fragile]
\begin{question}
Using the car data, test the hypothesis that the mean distance at each fill up is less than 450km.
\end{question}
\vfill
\begin{question}
Use the car data to see if the mean distance for 2015 fill ups is different than the mean distance for 2016 fill ups.
\end{question}
%\pause
%\pause
%\verb|t.test(x = car_data Distance, alternative=c("less"), mu=450)|
%\pause
%\verb|t.test(Distance~prov, data=car_data, alternative=c("two.sided"), mu=0, paired =FALSE)|
\end{frame}
%%%%%%%%%%%%%%%%%%%%%%%%%%%%%%%%%%%%%%%%%%%%%%%%%%%%%%%%%%%%%%%%%%%%%%%%%%%%%%%%%%%%%%%%%%%%%%%%%%%%


%%%%%%%%%%%%%%%%%%%%%%%%%%%%%%%%%%%%%%%%%%%%%%%%%%%%%%%%%%%%%%%%%%%%%%%%%%%%%%%%%%%%%%%%%%%%%%%%%%%%
\begin{frame}[fragile]{Linear models in R}
\vfill
A linear model is an equation that relates a response variable ($y$) to some explanatory variables ($x$'s).  The general form of the model is:
$$y = b_0 + b_1x_1 + b_2x_2 + \cdots + b_nx_n$$
\vfill
Not all of the data points can fall on this line so the full equation is
$$y_i = b_0 + b_1 x_{1i} + b_2x_{2i} + \cdot + b_nx_{bi} + \varepsilon_i$$
\vfill
Where $\varepsilon_i$ denotes the error term associated with observation $i$.
\vfill
\end{frame}
%%%%%%%%%%%%%%%%%%%%%%%%%%%%%%%%%%%%%%%%%%%%%%%%%%%%%%%%%%%%%%%%%%%%%%%%%%%%%%%%%%%%%%%%%%%%%%%%%%%%

%%%%%%%%%%%%%%%%%%%%%%%%%%%%%%%%%%%%%%%%%%%%%%%%%%%%%%%%%%%%%%%%%%%%%%%%%%%%%%%%%%%%%%%%%%%%%%%%%%%%
\begin{frame}[fragile]{Linear models in R}
We are assuming
\begin{enumerate}
\item Residuals are independent.
\item Residuals are normally distributed.
\item Residuals have a mean of 0 for all $X$.
\item Residuals have constant variance.
\end{enumerate}
\end{frame}
%%%%%%%%%%%%%%%%%%%%%%%%%%%%%%%%%%%%%%%%%%%%%%%%%%%%%%%%%%%%%%%%%%%%%%%%%%%%%%%%%%%%%%%%%%%%%%%%%%%%


%%%%%%%%%%%%%%%%%%%%%%%%%%%%%%%%%%%%%%%%%%%%%%%%%%%%%%%%%%%%%%%%%%%%%%%%%%%%%%%%%%%%%%%%%%%%%%%%%%%%
\begin{frame}[fragile]{Fitting a Linear Model}
\begin{Verbatim}[xleftmargin=2em, xrightmargin=1.5em, frame=single, label=Using P-value, framesep=0.5em, commandchars=\\\{\}, fontsize=\small]
> lm(km.L~Litres+Distance, data=car_data)
> model

call:
lm(formula = km.L ~ Litres + Distance, data = car_data)

Coefficients:
(Intercept)        Litres        Distance
  10.35447       -0.33295         0.03251
\end{Verbatim}
The formula can be then be created using the values stored in \verb|model$coefficients|

\verb|Km.L = 10.35447 -0.33295*Litres + 0.03251*Distance|
\end{frame}
%%%%%%%%%%%%%%%%%%%%%%%%%%%%%%%%%%%%%%%%%%%%%%%%%%%%%%%%%%%%%%%%%%%%%%%%%%%%%%%%%%%%%%%%%%%%%%%%%%%%


%%%%%%%%%%%%%%%%%%%%%%%%%%%%%%%%%%%%%%%%%%%%%%%%%%%%%%%%%%%%%%%%%%%%%%%%%%%%%%%%%%%%%%%%%%%%%%%%%%%%
\begin{frame}[fragile]{Conclusion}
\begin{enumerate}
\item R is a free and open source programming language for statistical computing and graphics.
\item R contains many useful features for data analysis including data structures such as vectors and data frames that make it easy to perform statistical analysis and visualization.
\item R is often used for hypothesis testing and understanding how to properly setup and interpret a test is an important skill.
\end{enumerate}
\end{frame}
%%%%%%%%%%%%%%%%%%%%%%%%%%%%%%%%%%%%%%%%%%%%%%%%%%%%%%%%%%%%%%%%%%%%%%%%%%%%%%%%%%%%%%%%%%%%%%%%%%%%



\end{document}
%%%%%%%%%%%%%%%%%%%%%%%%%%%%%%%%%%%%%%%%%%%%%%%%%%%%%%%%%%%%%%%%%%%%%%%%%%%%%%%%%%%%%%%%%%%%%%%%%%%%
