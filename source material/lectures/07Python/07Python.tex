\documentclass[xcolor=svgnames]{beamer}
%\documentclass[xcolor=svgnames, handout]{beamer}

%\includeonlyframes{current}

\usepackage[utf8]    {inputenc}
\usepackage[T1]      {fontenc}
\usepackage[english] {babel}

\hypersetup{
     colorlinks = true,
     linkcolor = blue,
     anchorcolor = blue,
     citecolor = blue,
     filecolor = blue,
     urlcolor = blue
     }
     
     
     
\usepackage{amsmath,amsfonts,graphicx}
\usepackage{beamerleanprogress}
\usepackage{xcolor}
\usepackage{soul}
%\usepackage{verbatim}
\usepackage{multicol}
\usepackage{tikz} 
\usepackage[export]{adjustbox}

\usepackage{fancyvrb}
\usepackage{spverbatim}

\usepackage{xcolor}


% https://tex.stackexchange.com/questions/182476/how-do-i-center-a-boxed-verbatim/182479
\newsavebox{\FVerbBox}
\newenvironment{FVerbatim}
 {\VerbatimEnvironment
  \begin{center}
  \begin{lrbox}{\FVerbBox}
  \begin{BVerbatim}}
 {\end{BVerbatim}
  \end{lrbox}
  \fbox{\usebox{\FVerbBox}}
  \end{center}}
  
  
  
\definecolor{iyellow}{RGB}{255, 162, 23}
\definecolor{sgreen}{RGB}{118, 191, 138}

\newcommand{\yellow}[1]{\textcolor{iyellow}{#1}}
\newcommand{\red}[1]{\textcolor{red}{#1}}
\newcommand{\green}[1]{\textcolor{ForestGreen}{#1}}
\newcommand{\blue}[1]{{\textcolor{blue}{#1}}}
\newcommand{\orange}[1]{{\textcolor{orange}{#1}}}
\newcommand{\bblue}[1]{\textcolor{SteelBlue!90!gray}{#1}} % beamer blue
\newcommand{\purple}[1]{{\textcolor{purple}{#1}}}


\newcommand{\el}{\\[1em]\pause}
\newcommand{\nl}{\\[1em]}
\newcommand{\define}[1]{\textbf{\textcolor{orange}{#1}}}
\newcommand{\answer}[1]{\textit{\textbf{\textcolor{iyellow}{#1}}}}
\newcommand{\command}[1]{\texttt{\textbf{\textcolor{DarkMagenta}{#1}}}}
\newcommand{\ipic}[2]{\includegraphics[width={#2}\textwidth]{#1}}
\newcommand{\cell}[1]{{\sf \textbf{\textcolor{DarkMagenta}{#1}}}}
\newcommand{\ra}{$\rightarrow$}

% timed answer
\newcommand{\tans}[2]{\textbf<#1>{\textit<#1>{{\color<#1>{iyellow}{#2}}}}}

\newcommand{\ft}[1]{\frametitle{#1}}

% for straight quotes in verbatim:
\usepackage{upquote,textcomp}

\usepackage[T1]{fontenc}
\usepackage[utf8]{inputenc}
\usepackage{tikz}
\usetikzlibrary{shadows}

\usepackage{upquote,textcomp}
\newenvironment{allintypewriter}{\ttfamily}{\par}
\newcommand{\bs}{$\backslash$}




\newcommand*\keystroke[1]{%
  \tikz[baseline=(key.base)]
    \node[%
      draw,
      fill=white,
      drop shadow={shadow xshift=0.25ex,shadow yshift=-0.25ex,fill=black,opacity=0.75},
      rectangle,
      rounded corners=2pt,
      inner sep=1pt,
      line width=0.5pt,
      font=\scriptsize\sffamily
    ](key) {#1\strut}
  ;
}

\title
  [Data 301 Data Analytics]
  {Data 301 Data Analytics\\
Reading and Writing Files in Python}

\author
  [Dr.\ Irene Vrbik]
  {Dr.\ Irene Vrbik}

\date
  {}

\institute
  {University of British Columbia Okanagan \newline irene.vrbik@ubc.ca}


\graphicspath{{img/}}

\begin{document}

\maketitle

\setbeamersize{description width=0.57cm} % to have less indent with the description environment



\begin{frame}{Python File Input/Output}
Many data processing tasks require reading and writing to files.\nl

\command{open()} is a built-in function for creating, writing and reading files.\nl

This function takes two parameters; \textit{filename}, and \textit{mode}\footnote{there are other optional paramters, see {\tt help(open)}}.\nl

\begin{itemize}
\item Open a file for \emph{reading} (default):\\
\begin{allintypewriter}
infile = open("input.txt", mode = "\purple{\bf r}")
\end{allintypewriter}
\item  Open a file for \emph{writing}:\\
\begin{allintypewriter}
myfile = open("output.txt",  mode = "\purple{\bf w}")
\end{allintypewriter}
%\item Open a file for \emph{read/write}:\\
%\begin{allintypewriter}
%myfile = open("data.txt",  mode =  "\purple{\bf r+}")\nl
%\end{allintypewriter}
\end{itemize}
\vfill
Python will look for the file in the current directory, but you can specify an path if it is located elsewhere.
\vfill
\end{frame}


\begin{frame}[fragile]\ft{Reading from a text file (as one string)}
\begin{itemize}
%\item Note that if {\tt data.txt"} is not in your current directory, you can specify its path. \nl

\item Notice that even if you are reading in a text file, the object {\tt myfile} will \textit{not} be a string.
\end{itemize}
\begin{Verbatim}[frame=single, xleftmargin=0.5in]
>>> infile = open("input.txt", "r") 
>>> class(infile)
<class '_io.TextIOWrapper'>
\end{Verbatim}
\begin{itemize}
\item The \command{open} function returns a  \href{https://docs.python.org/3/glossary.html#term-file-object}{file object}.\nl
\item In the above, I saved my file object to a variable called \command{infile}.  Common naming conventions for file input/file output objects are \command{fin} and \command{fout}, resp.\nl
\item N.B. If the file cannot be opened, an {\tt OSError} is raised.\nl

\end{itemize}
\end{frame}

\begin{frame}[fragile]\ft{Reading from a text file (as one string)}
\begin{itemize}
\item One way to read a text file in Python is using the {\tt <filename>.read()} method:\nl
\item By default \command{read()} will return the entire text document.
\begin{FVerbatim}
>>> this = infile.read()
>>> print(this)
'1\n2\n3\n4\n5\n6\n7\n8\n9\n10\n'
\end{FVerbatim}
\item Notice that all characters (including line breaks \verb|\n|) are returned.\nl
\item The out outputs of the \command{read()} method is a string
\begin{FVerbatim}
>>> type(this)
<class 'str'>
\end{FVerbatim}
\end{itemize}
\end{frame}

\begin{frame}[fragile]\ft{Reading from a text file by parts}
\begin{itemize}
\item Alternatively we can specify the the number characters you want to return as an argument (notice how newline character \verb|\n| is treated as a character:
\begin{FVerbatim}
>>> infile = open("input.txt", "r")
>>> infile.read(4)
'1\n2\n'
\end{FVerbatim}
\item \command{read} will keep our place in the document so that if we call it again it will read the \textit{next} 4 characters.  % until there are no more to be read
\begin{FVerbatim}
>>> infile.read(4)
'3\n4\n'
\end{FVerbatim}
\item Notice how when we call these strings in the \command{print()} function \verb!\n! is converted to new lines:
\begin{FVerbatim}
>>> print(infile.read(4))
5
6
\end{FVerbatim}
\end{itemize}
\end{frame}


\begin{frame}[fragile]\ft{Reading from a text file by parts}
We can repeatedly call this function until there are no more characters to be read.
\begin{Verbatim}[commandchars=\\\{\}, xleftmargin=0.5in, frame=single]
\textcolor{OliveDrab}{# this reaches the end of the document}
>>> print(infile.read(9))
7
8
9
10

\textcolor{OliveDrab}{# there are no more characters to be read}
>>> infile.read(1) 
''
\end{Verbatim}

\end{frame}



\begin{frame}[fragile]\ft{Reading from a text file by line}
\begin{itemize}
\item Alternatively we can read the contents line by line:
\begin{FVerbatim}
>>> infile = open("input.txt", "r")
>>> infile.readline()
'1\n'
>>> infile.readline()
'2\n'
\end{FVerbatim}
\item As before, it will keep track of where we are in the file until there are no more lines to be read:
\begin{FVerbatim}
# ...
>>> infile.readline()
'9\n'
>>> infile.readline()
'10\n'
>>> infile.readline()
''
\end{FVerbatim}
\end{itemize}
\end{frame}



\begin{frame}[fragile]\ft{Reading from a text file (as one string)}
It is good practice to close a file once you no longer need it to free up resources.  Once the file is closed, any further attempts to use the file object ({\tt infile}) will fail. \vfill
 \begin{Verbatim}[commandchars=\\\{\}, frame=single]
>>> fin = open("input.txt", "r") 
>>> print(fin.read(5))
1
2
3

>>> fin.close() \textcolor{OliveDrab}{ # close the file}
>>> fin.read(5)  \textcolor{OliveDrab}{# can no longer read from file}
\textcolor{red}{Traceback (most recent call last):}
\textcolor{red}{  File "<stdin>", line 1, in <module>}
\textcolor{red}{ValueError: I/O operation on closed file.}
\end{Verbatim}
\vfill
\end{frame}

\begin{frame}[fragile]\ft{Reading from a text file (as one string)}
We have the option to save the entire file as a string object to a variable. If we do that, we can access the information regardless of whether or not the file object has been closed:
\vfill
 \begin{Verbatim}[commandchars=\\\{\}, frame=single]
fin = open("input.txt", "r") 
val = fin.read() \textcolor{OliveDrab}{ # read file as one string}
fin.close()\textcolor{OliveDrab}{       # close the file}
print(val)  \textcolor{OliveDrab}{      # the entire contents of the file}
print(val[0:4])  \textcolor{OliveDrab}{ # first 4 characters}
print(val[4:8])  \textcolor{OliveDrab}{ # next 4 characters}
print(fin.closed) \textcolor{OliveDrab}{# returns True}
\end{Verbatim}
\vfill
\begin{block}{}
You can check whether the file is close using {\tt <filename>.closed} which returns  {\tt True} or {\tt False}.
\end{block}
\vfill
\end{frame}

\begin{frame}[fragile]{Reading from a text file into a list}
\begin{itemize}
\vfill
\item We could read \textit{every} line using the \command{readlines()} method (as opposed to \command{readline()} (singular which reads a single line).
\vfill
\item Notice that \command{readlines()} returns a Python list which includes the end of line characters.
\end{itemize}
\vfill
\begin{Verbatim}[fontsize=\small, frame=single]
>>> fin = open("input.txt", "r")
>>> alllines = fin.readlines()
>>> print(alllines) 
['1\n', '2\n', '3\n', '4\n', '5\n', '6\n', '7\n', '8\n', 
'9\n', '10\n']
\end{Verbatim}
\vfill
\end{frame}


\begin{frame}[fragile]\ft{Iteration with files}
As with any Python list, we can iterate through its contents element by element  using a \command{for} loop.  \nl

For this particular scenario, this corresponds to going through the {\tt input.txt} file line by line.\nl
\begin{Verbatim}[xleftmargin=.5in, frame=single]
>>> for lines in alllines:
...     print(lines)
... 
1

2

...

10
\end{Verbatim}
\vfill
\end{frame}

\begin{frame}[fragile]\ft{Iteration with files}
You may have noticed  extra whitespace between lines.\nl
This is because the  \command{print} function has a newline character printed by default (eg, we are actually printing \verb|'1\n\n'| for the first line).\nl
To remedy this we could use the {\command{strip}} method to remove the newline characters before printing.\nl
\href{https://www.geeksforgeeks.org/python-string-strip-2/}{strip()} removes any characters specified as arguments from both left and right of a string.  If no argument is specified, then all whitespace from starting from the left (resp. right) is removed until we reach the first non-match.
\begin{Verbatim}[frame=single, fontsize=\small, commandchars=\\\{\}]
>>> '\textcolor{red}{aaca}bb\textcolor{blue}{aacc}bb\textcolor{red}{cacc}'.strip('ac') \textcolor{red}{# removed}/\textcolor{blue}{not removed}
'bbaaccbb'
\end{Verbatim}
\end{frame}

\begin{frame}[fragile]\ft{Iteration with files}
Hence we can iterate through this file line by line using:
\begin{Verbatim}[xleftmargin=.5in, frame=single]
>>> for lines in alllines:
...     print(lines.strip('\n'))
... 
1
2
...
10
\end{Verbatim}
\vfill
While this worked well for this small file, if our text file was very long, creating this list would consume a lot of memory.  
\vfill
To get around this, we can loop over the file line by line, and only read the lines of text that your program needs\dots
\end{frame}


%\begin{frame}[fragile]\ft{Other File Methods}
%We can read all of the lines at once using the {\tt readlines()} method.  Unlike the  {\tt read()} method, this saves all the lines of the file into a list:
%\begin{Verbatim}[xleftmargin=.5in, frame=single]
%infile = open("input.txt", "r")
%
%# Read all lines in the file into a list
%lines = infile.readlines()
%infile.close()
%print(infile.closed)	# True
%# lines will still be available
%\end{Verbatim}
%\begin{Verbatim}[frame=single, fontsize=\small]
%print(lines)
%# ['1\n', '2\n', '3\n', '4\n', '5\n', '6\n', '7\n', '8\n', '9\n', '10\n']
%\end{Verbatim}
%\end{frame}



\begin{frame}\ft{Iteration with files}
\begin{itemize}
\item The \emph{file objects} themselves are actually iteratable (that is we can iterate through them in a \command{for} loop).\nl
\item Hence we traverse through the text file line by line and do something with it, eg, print the contents to the screen.\nl
\item Remember that the end of line characters are part of the string on each line. \nl
\item We may choose to remove them in combination with print so that we don't produce all that unnecessary white space.  \nl
\item As before, we mustn't forget to close the file once we are done with it.\nl
\end{itemize}
\end{frame}





\begin{frame}[fragile]\ft{Iteration with files}
By using a loop we can iterate through the whole file line by line:
\begin{FVerbatim}
infile = open("input.txt", "r")
for x in infile:
  print(x)
\end{FVerbatim}
If we only wanted to read this file until we say a \command{4}, say, we could exit the \command{for} loop using \command{break}:\nl
\begin{columns}
\begin{column}{0.5\textwidth}
\begin{Verbatim}[frame=single, fontsize=\small, label=check for equality]
fin = open("input.txt", "r")
for x in fin:
    line = x.strip('\n')
    print(line)
    if (int(line) == 4):
        break
fin.close()
\end{Verbatim}
\end{column}
\begin{column}{0.5\textwidth}  %%<--- here
\begin{Verbatim}[frame=single, fontsize=\small, label=check for substring]
fin = open("input.txt", "r")
for x in fin:
    print(x.strip('\n'))
    if '4' in x:
        break
fin.close()
\end{Verbatim}
\end{column}
\end{columns}
\end{frame}







%\begin{frame}[fragile]\ft{Reading text from a file line by line}
%
%The {\tt <filename>.readline()} function reads a single line of the file.  Every time you call it, it will move to the next line.
%
%\begin{Verbatim}[xleftmargin=.5in]
%infile = open("input.txt", "r")
%print(infile.readline()) # prints the first line
%print(infile.readline()) # prints the second line
%\end{Verbatim}
%
%Alternatively, we could read every line using the {\tt <filename>.readlines()} function
%\begin{Verbatim}[xleftmargin=.5in]
%infile = open("input.txt", "r")
%print(infile.readlines()) # not ideal way to view
%\end{Verbatim}
%Notice that {\tt readlines()} includes the end of line characters.
%\end{frame}

%
%\begin{frame}[fragile]\ft{Reading text from a file line by line}
%The {\tt read} functions will keep track of where it is in the file and pick up where it left off.  When it reaches the end of the file, {\tt read} will just return an empty string.\nl
%
%To locate where you are in the file, you can use
%\begin{Verbatim}[xleftmargin=.5in]
%infile = open("input.txt", "r") 
%print(infile.read(5)) # prints first 5 characters
%print(infile.read(5)) # prints the next 5 characters
%
%# see our current position in the file
%print(infile.tell()) # returns 10
%infile.close()
%
%\end{Verbatim}
%% f.seek() will set your position: f.seek(0) will bring you back to the start of the file
%\end{frame}



\begin{frame}[fragile]\ft{Reading text from a file line by line}
%To cycle through the lines of a text document we can use a loop.  This will only read one line at a time.\nl
%We could get rid of that extra newline character using strip:
It is worth mentioning that we can also cycle through the lines of text document using a \command{while} loop:
%%infile = open("input.txt", "r")
%for line in infile:
%     print(line.strip('\n'))
%infile.close()
\vfill
\begin{Verbatim}[frame=single, label = Using a for loop]
infile = open("input.txt", "r")
for line in infile:
    print(line.strip('\n'))
infile.close()
\end{Verbatim}
\vfill
\begin{Verbatim}[frame=single, label = Using a while loop]
infile = open("input.txt", "r")
line = infile.readline()
while line != "":
    line = infile.readline()
    print(line.strip('\n'))
infile.close()
\end{Verbatim}
\vfill
\end{frame}

\begin{frame}[fragile]\ft{Reading text from a file line by line}
To avoid writing programs which forget to close the file, we could also use \href{https://effbot.org/zone/python-with-statement.htm}{\tt with}\footnote{that link may make more sense after we cover {\tt try-except} statements}.  The \command{with} statement will automatically close the file after the suite is exited.  Hence we never have to write {\tt infile.close()}.\nl
\begin{Verbatim}[xleftmargin=.5in, frame=single]
# The following will auto-close file
with open("input.txt", "r") as infile:
     for line in infile:
          print(line.strip('\n'))
print(infile.close()) # returns True
\end{Verbatim}
\begin{block}{Tip:}
The {\tt with} option is considered best practice as it automatically closes the file for us.
\end{block}

\end{frame}





\begin{frame}[fragile]\ft{Writing to a Text File}
\vfill
Selecting the write (\verb|'w'|) mode will allow us to \emph{write} text to a file
\vfill
\begin{Verbatim}[frame=single]
outfile = open("output.txt", "w")
# writes the numbers 1 through 10 on new lines
for n in range(1,11):
    outfile.write(str(n) + "\n")
# not written to final until we run the following:
outfile.close() 
\end{Verbatim}
\vfill
\begin{alertblock}{Warnings}
\begin{enumerate}
\item Python will try to overwrite the file {\tt output.txt} if it exists, otherwise, the file will be created.
\item The contents are not written to file until we close it. 
\item Numbers need to be converted to strings before writing.
\end{enumerate}
\end{alertblock}
\vfill
\end{frame}

\begin{frame}[fragile]\ft{Writing to a Text File}
The second warning outlined on the previous slide is yet another reason that the {\tt with} method is generally preferred. %N.B that the file (with I know call {\tt f} instead of {\tt outfile} will close automatically.
\begin{Verbatim}[xleftmargin=.5in, frame=single]
with open('output.txt', 'w') as fout:
    for n in range(1,11):
        fout.write(str(n) + "\n")
\end{Verbatim}
Notice how this method makes it impossible for us to forget to close a file.
\end{frame}



\begin{frame}[fragile]\ft{Writing to a Text File}
 This will overwrite the file if it already exists, otherwise, it, creates a new file for writing.  The following will overwrite the contents of {\tt output.txt}:
\begin{Verbatim}[xleftmargin=.5in, frame=single]
with open("output.txt", "w") as f:
    f.write("Test")
\end{Verbatim}
\vfill
To create an empty file, we can use the \command{pass} command 
\vfill
\begin{Verbatim}[xleftmargin=.5in, frame=single]
# creates an empty file
with open("test2.txt", "w") as f:
    pass
\end{Verbatim}
Another alternative for creating an empty file is:
\vfill
\begin{Verbatim}[xleftmargin=.5in, frame=single]
open(filename, 'w').close()
\end{Verbatim}
\vfill
\end{frame}


\begin{frame}[fragile]\ft{Writing to a Text File}
Like the \command{read} functions, \command{write} will remember its place within the document and will pick up where it left off:
\vfill
\begin{Verbatim}[xleftmargin=.5in, frame=single]
with open("test.txt", "w") as fout:
    fout.write("Test")
    fout.write("Test again")
\end{Verbatim}
Test.txt file will contain:
\begin{quote}
TestTest again
\end{quote}
To include line breaks, we need to include the newline character \verb|\n|.
\begin{Verbatim}[xleftmargin=.5in, frame=single]
with open("test.txt", "w") as f:
    f.write("Test")
    f.write("\n Test again")
\end{Verbatim}
Test.txt file will contain:
\begin{quote}
Test\\
Test again
\end{quote}
\end{frame}



\begin{frame}[fragile]\ft{Writing to a Text File}
Once we close the file however, we need to select the \emph{append} mode (\verb|'|\command{a}\verb|'|)  in order to add text to the end of the document. 
\begin{Verbatim}[xleftmargin=.5in, frame=single]
outfile = open("output.txt", "a")
for n in range(11,20):
    outfile.write(str(n) + "\n")
outfile.close() 
\end{Verbatim}
\begin{itemize}
\item As in the \command{w} (write) mode, the contents are not written to file until we close the file.
\item  Note that, if the file does not exist, it creates a new file for writing. 
\end{itemize}
\end{frame}





%\begin{frame}[fragile]\ft{}
%\begin{Verbatim}[commandchars=\\\{\}]
%\textcolor{brown}{test} this
%\end{Verbatim}
%\end{frame}


\begin{frame}[fragile]\ft{Using {\tt split} for CSV files}
A common type of file you may want to read into your program is a comma separated value (CSV) file.\nl
We can read csv files by iterating over the file object and using \command{strip} and \command{split}:\nl
\begin{Verbatim}[xleftmargin=.5in, frame=single]
with open("data.csv", "r") as infile:
    for line in infile:
        line = line.strip(" \n")
        fields = line.split(",")
        for i in range(0,len(fields)):
            fields[i] = fields[i].strip()
        print(fields)
\end{Verbatim}
\end{frame}


\begin{frame}[fragile]\ft{Using {\tt split} for CSV files}
\vfill
In the code from the previous slide:
\begin{itemize}
\item \command{line} is a string (with the end of line character \verb|\n| removed)
\item \command{fields} is a list containing the individual cell values for the corresponding row.
\end{itemize}
\vfill
For the last row in our CSV file:
\vfill
\begin{Verbatim}[frame=single]
>>> print(type(line))
<class 'str'>
>>> print(line)
1.7702,1.1211,-0.6032,-0.6982,0.4066
>>> print(type(fields))
<class 'list'>
>>> print(fields)
['1.7702', '1.1211', '-0.6032', '-0.6982', '0.4066']
\end{Verbatim}
\vfill
\end{frame}



\begin{frame}[fragile]\ft{Using modules: {\tt csv} for CSV files}
Alternatively, you can  use the \command{csv} module to read csv files:\nl

By importing the module named {\tt csv}, we can now call the {\tt csv.reader} function (see more \href{https://docs.python.org/3/library/csv.html}{\bf here}).\nl

A useful module/function in case you forget the name of your file is the \command{os.listdir()} to list all files in a directory.  
\begin{Verbatim}[xleftmargin=.5in, frame=single]
import os
print(os.listdir("."))
\end{Verbatim}
We can also use the module {\tt pprint} (for pretty print) to make this output a little neater:
\begin{Verbatim}[xleftmargin=.5in, frame=single]
from pprint import pprint
pprint(os.listdir("."))
\end{Verbatim}
\end{frame}


% https://realpython.com/python-csv/
\begin{frame}[fragile]\ft{Using modules: {\tt csv} for CSV files}
%A module is a file containing a set of functions.  
\begin{Verbatim}[xleftmargin=.1in,commandchars=\\\{\}, frame=single]
\purple{import csv}
with open("data.csv", "r") as infile:
	\purple{csvfile = csv.reader(infile)}
	for row in csvfile:
		print(row)
\end{Verbatim}
\begin{itemize}
\item \command{csvfile} is a \emph{reader object} we can iterate over in a for loop.
\item Each iteration corresponds to a line from {\tt data.csv}.  
\item Each \command{row} is a Python list of \emph{string} elements containing the data found by removing the delimiters.
\end{itemize}

%N.B. you can create your own  module under the file extension .py
\begin{Verbatim}[frame=single, fontsize=\small]
>>> print(type(csvfile))
<class '_csv.reader'>
>>> print(type(row))
<class 'list'>
>>> print(row)
['1.7702', '1.1211', '-0.6032', '-0.6982', '0.4066']
\end{Verbatim}
\end{frame}


\begin{frame}[fragile]\ft{Using modules: {\tt csv} for CSV files}
Remember that each element in this list is currently being treated as a string.  Before we do any calculations on this numeric values, we need to convert them using \command{float}.\nl

\begin{Verbatim}[xleftmargin=.1in,commandchars=\\\{\}, frame=single]
\purple{import csv}
# only print the rows that start with a number> 1 
with open("data.csv", "r") as infile:
	\purple{csvfile = csv.reader(infile)}
	for row in csvfile:
		if float(row[0]) > 1:
			print(row)
\end{Verbatim}
\end{frame}






%%%%%%%%%%%%%%%%%%%%%%%%%

% question:

\begin{frame}[fragile]\ft{}
  \begin{example}
How many of the following statements are TRUE?
\begin{enumerate}
\item {{A Python file is automatically closed for you.
}}
\item {{If you use the {\tt with} syntax, Python will close the file for you.
}}
\item {{To read from a file, use {\tt w} when opening a file.
}}
\item {{The {\tt read()} method will read the entire file into a string.
}} 
\item {{You can use a for loop to iterate through all lines in a file.}} 
\end{enumerate}
\begin{multicols}{5}
\begin{enumerate}[A)]
\item 0 
\item 1
\item 2
\item 3
\item 4
\end{enumerate}
\end{multicols}
  \end{example} 
\end{frame}


% answer:

\begin{frame}<handout:0>[fragile]\ft{}
  \begin{block}{Answer:}
How many of the following statements are TRUE?
\begin{enumerate}
\item {\color<1->{red}	{A Python file is automatically closed for you.
}}
\item {\color<2->{sgreen}	{If you use the {\tt with} syntax, Python will close the file for you.
}}
\item {\color<3->{red}	{To read from a file, use {\tt w} when opening a file.
}}
\item {\color<4->{sgreen}	{The {\tt read()} method will read the entire file into a string.
}} 
\item {\color<5->{sgreen}	{You can use a for loop to iterate through all lines in a file.}} 
\end{enumerate}
\begin{multicols}{5}
\begin{enumerate}[A)]
\item 0 
\item 1
\item 2
\item \tans{5}{3} 
\item 4
\end{enumerate}
\end{multicols}
  \end{block} 
\end{frame}


%%%%%%%%%%%%%%%%%%%%%%%%%




\begin{frame}\ft{Try it}
\begin{example}
\begin{enumerate}
\item  Write a Python program that writes to the file test.txt the numbers from 20 to 10 on its own line in descending order.

\item Write a Python program that reads your newly created test.txt file line by line and only prints out the value if it is even.

\item Print out the contents of the census file  \href{https://www150.statcan.gc.ca/t1/tbl1/en/tv.action?pid=1710000501}{provinces.csv} available on Canvas \label{provinces} (You may use the {\tt csv} module).

\item Try to print out only the provinces with population$ > 1 $ million people in 2015 from the data in \ref{provinces}.  Hint: You will need to remove the commas from the numbers (eg. 44214 instead of 44,214) using the {\tt replace()} function.

\end{enumerate}
\end{example}
\end{frame}


%%%%%%%%%%%%%%%%%%%%%%%%%

\begin{frame}\ft{Handling Errors and Exceptions}
An \define{exception} is an error situation that must be handled or the program will fail. \nl
\define{Exception handling} is how your program deals with these errors.
Examples:

\begin{description}
\item[ZeroDivisionError] Attempting to divide by zero
\item[IndexError] An array index that is out of bounds
\item[TypeError] operation is applied to an object of an incorrect type.
\item[NameError]  when an object could not be found
\item[SyntaxError] when a syntax error is encountered
%\item A specified file that could not be found
%\item A requested I/O operation that could not be completed normally
%\item Attempting to follow a null or invalid reference
%\item Attempting to execute an operation that violates some kind of security measure
\end{description}
See a list of all errors \href{https://www.tutorialsteacher.com/python/error-types-in-python}{here}.
\end{frame}

\begin{frame}[fragile]
Example taken from \href{https://docs.python.org/3/tutorial/errors.html}{here}\\
\begin{allintypewriter} { \small
$>>>$ 10 * (1/0)\newline
Traceback (most recent call last):\newline
  File "<stdin>", line 1, in <module>\newline
\red{ZeroDivisionError}: division by zero\newline
$>>>$ 4 + spam*3\newline
Traceback (most recent call last):\newline
  File "<stdin>", line 1, in <module>\newline
\red{NameError}: name 'spam' is not defined\newline
$>>>$ '2' + 2\newline
Traceback (most recent call last):\newline
  File "<stdin>", line 1, in <module>\newline
\red{TypeError}: Can't convert 'int' object to str implicitly}
\end{allintypewriter}

\end{frame}


\begin{frame}\ft{The {\tt try-except} statement}
The \command{try-except} statement will handle an exception that may occur in a  block of statements:\nl

Execution flow:
\begin{itemize}
\item The statements in the try block are executed.
\item If no exception occurs:
\begin{itemize}
\item If there is an else clause, it is executed.
\item Continue on with next statement after try.
\end{itemize}
\item If an exception occurs:
\begin{itemize}
\item Execute the code after the except.
\end{itemize}
\item If the optional \command{finally} block is present, it is always executed regardless if there is an  exception or not.
\item Keyword \command{pass} is used if any block has no statements.
\end{itemize}

For more information see \href{https://docs.python.org/3/tutorial/errors.html}{click here}
\end{frame}
%References: https://docs.python.org/3/tutorial/errors.html
%
%Notes:
%May list multiple exceptions as a tuple: except (RuntimeError, TypeError, NameError):
%May catch all exceptions by just putting except:

\begin{frame}[fragile]\ft{Python Exceptions Block}
The general syntax is 
\begin{Verbatim}[xleftmargin=0.1in, commandchars=\\\{\}, frame=single] 
\green{try:}
    \green{# try something that may produce an exception}
\purple{except ErrorName:}
    \purple{# only executed if an ErrorName exception}
    \purple{# is raised above}
# the following lines (else and finally clauses) 
# are optional:
\blue{else:}
    \blue{# only executed if no exception}
\red{finally:}
    \red{# always executed}
\end{Verbatim}
{\bf N.B.} A try statement may have more than one except clause, to specify handlers for different exceptions. However, at most one handler will be executed.
\end{frame}
%%%%%%%%%%%%%%%%%%%%%%%%%


\begin{frame}[fragile]
This could be useful in the context of reading files: If we try to read a file that does not exist, we need not have our entire program fail, and try to catch this exception in the following manner:
\begin{Verbatim}[frame=single, xleftmargin=0.5in]
filename = 'nonexistingfile.txt'
try:
    with open(filename, 'r') as f:
        reader = csv.reader(f)
        for row in reader:
            pass #do stuff here
except FileNotFoundError:
    print("Could not read file:", filename)
\end{Verbatim}
If the {\sf nonexistingfile.txt} doesn't live in our current working directory the above produces: \verb|Could not read file: nonexistingfile.txt|.
\end{frame}







\begin{frame}[fragile]\ft{Python Exceptions Block}

\begin{Verbatim}[xleftmargin=.5in, commandchars=\\\{\}] 
\green{try:}
    \green{# try block, exit if error}
    \green{num = int(input("Enter a number:"))}
    \green{print("You entered:",num)}
\purple{except ValueError:}
    \purple{# only executed if exception}
    \purple{print("Error: Invalid number")}
\blue{else:}
    \blue{# only executed if no exception}
    \blue{print("Thank you for the number")}
\red{finally:}
    \red{# always executed}
    \red{print("Always do finally block")}
\end{Verbatim}
If the input has not been a valid integer, we will generate (raise) a {\tt ValueError}.
\end{frame}




\begin{frame}[fragile]\ft{Python Exceptions Block}
We can be more generic and catch any error using the following
\begin{Verbatim}[xleftmargin=0.1in, commandchars=\\\{\}] 
\green{try:}
    \green{# try something that may produce an exception}
\purple{except:}
    \purple{# only executed if an exception is raised above}
\blue{else:}
    \blue{# only executed if no exception}
\red{finally:}
    \red{# always executed}
\end{Verbatim}
\end{frame}
%%%%%%%%%%%%%%%%%%%%%%%%%

% question:

\begin{frame}[fragile]\ft{}
  \begin{example}
  What is the output of the following code if we enter 10?
\begin{Verbatim}
try:
    num = int(input("Enter an integer:"))
    print(num)
except ValueError:
    print("Invalid")
else:
    print("Thanks")
finally:
    print("Finally")
\end{Verbatim}
\begin{multicols}{5}
\begin{enumerate}[A)]
\item 10 \newline\newline
\item 10\newline Finally \newline
\item Invalid\newline\newline
\item 10 \newline Thanks \newline
\item 10 \newline Thanks \newline Finally
\end{enumerate}
\end{multicols}
  \end{example} 
\end{frame}


% answer:

\begin{frame}<handout:0>[fragile]\ft{}
  \begin{block}{Answer:}
  What is the output of the following code if we enter 10?
\begin{Verbatim}
try:
    num = int(input("Enter an integer"))
    print(num)
except ValueError:
    print("Invalid")
else:
    print("Thanks")
finally:
    print("Finally")
\end{Verbatim}
\begin{multicols}{5}
\begin{enumerate}[A)]
\item 10 \newline\newline
\item 10 \newline Finally \newline
\item Invalid\newline\newline
\item 10 \newline Thanks \newline
\item \answer{10} \newline \answer{Thanks} \newline \answer{Finally}
\end{enumerate}
\end{multicols}
  \end{block} 
\end{frame}




\begin{frame}[fragile]\ft{}
  \begin{example}
  What is the output of the following code if we enter "{\tt hat}"?
\begin{Verbatim}
try:
    num = int(input("Enter an integer"))
    print(num)
except ValueError:
    print("Invalid")
else:
    print("Thanks")
finally:
    print("Finally")
\end{Verbatim}
\begin{multicols}{5}
\begin{enumerate}[A)]
\item hat \newline\newline
\item Invalid\newline  \newline
\item Invalid\newline Finally \newline
\item hat \newline Thanks \newline Finally
\item Finally \newline  \newline 
\end{enumerate}
\end{multicols}
  \end{example} 
\end{frame}


% answer:

\begin{frame}<handout:0>[fragile]\ft{}
  \begin{block}{Answer:}
  What is the output of the following code if we enter "{\tt hat}"?
\begin{Verbatim}
try:
	num = int(input("Enter num:"))
	print(num)
except ValueError:
    print("Invalid")
else:
    print("Thanks")
finally:
    print("Finally")
\end{Verbatim}
\begin{multicols}{5}
\begin{enumerate}[A)]
\item hat \newline\newline
\item Invalid\newline  \newline
\item \answer{Invalid}\newline \answer{Finally} \newline
\item hat \newline Thanks \newline Fianlly
\item Finally \newline  \newline 
\end{enumerate}
\end{multicols}
  \end{block} 
\end{frame}


%%%%%%%%%%%%%%%%%%%%%%%%%

\begin{frame}[fragile]{Raising Errors}
\begin{itemize}
\item Note that we can always generate an  error using the {\tt raise} statement.\nl
\begin{FVerbatim}
def raiseHell():
  try:
    raise ValueError
  except ValueError:
    print("You raised Hell!")
\end{FVerbatim}
Calling this function produces:
\begin{FVerbatim}
>>> raiseHell()
You raised Hell!
\end{FVerbatim}
\end{itemize}
\end{frame}




\begin{frame}\ft{Try it: Python Exceptions}
\begin{example}
 Write a Python program that reads two numbers and converts them to integers, prints both numbers, and then divides the first number by the second number and prints the result. 
\begin{itemize}
\item If we get an exception {\tt ValueError} when converting to an integer, print {\tt Invalid}.
\item If we get a {\tt ZeroDivisionError}, print {\tt Cannot divide by 0!}
\end{itemize}
\end{example}
\end{frame}




\begin{frame}\ft{Internet Terminology}
\vfill
%https://www.webopedia.com/DidYouKnow/Internet/ipv6_ipv4_difference.html
An \define{Internet Protocol (IP)} address is an identifier for %a computer on the Internet.
any device -- including computers, smartphones and game consoles -- that connects to the Internet.% requires an address.
\vfill
\begin{itemize}
\item IP version 4 (IPv4) address comprise 32 bits: 4 numbers in the range of 0 to 255.  The numbers are separated by dots. Eg: 
$${\tt 142.255.0.1}$$
\vfill
\end{itemize}
While IPv4 accommodates over 4 billion addresses, the number of unused IPv4 addresses will eventually run out.  For that reason, Internet Protocol version 6 (IPv6) is being deployed %as a successor to IPv4 to fulfill the need for more Internet addresses.
(which can accommodate 340 billion billion billion billion, addresses)
\vfill
\begin{itemize}
\item IP version 6 (IPv6) address have 128 bits and are % 16 numbers from 0 to 255 
represented as a series of eight 4-character hexadecimal numbers.  Eg: 
$${\tt 2002:CE57:25A2:0000:0000:0000:CE57:25A2}$$
% https://quizlet.com/158770659/cosc-122-internet-flash-cards/
\vfill
\end{itemize}
\end{frame}


\begin{frame}\ft{Internet Terminology}
\vfill
 A \define{domain} is a related group of networked computers. A \define{domain name} is a text name for computer(s) that are easier to remember. 
\begin{itemize}
\vfill
\item Eg. \emph{facebook.com} is the domain name for the IP address {\tt 31.13.80.36}
\vfill
\item Domain names are organized \emph{hierarchically}.  The most general part of the hierarchy is at the end of the name. 
\vfill
\item Example:  people.ok.ubc.ca:
ca = Canadian domain, ubc = University of British Columbia, ok = Okanagan campus, people = name of computer/server on campus 
\vfill
\end{itemize}
Read more about this in this three part series \href{https://www.spiria.com/en/blog/internet/understanding-the-internet-part-1-ip-addressing-and-ports/}{part 1}, \href{https://www.spiria.com/en/blog/internet/understanding-the-internet-part-2-domain-names/}{part 2}, \href{https://www.spiria.com/en/blog/internet/understanding-the-internet-part-3-practical-applications/}{part 3}.
\vfill
\end{frame}

%% https://whatismyipaddress.com/domain-name
% There are a lot of us on the internet. No two of us can have the same number.

% If the Internet is like a phone book, and a web page is like a physical building, the URL would be the precise street address of that building.  The IP address would be like the car that travels to its destination.

% 	The universal resource locator, or URL, is an entire set of directions, and it contains extremely detailed information. 
% The domain name is one of the pieces inside of a URL.  It is also the most easily recognized part of the entire address.

% https://www.networkworld.com/article/3268449/what-is-dns-and-how-does-it-work.html

% https://www.spiria.com/en/blog/internet/understanding-the-internet-part-1-ip-addressing-and-ports/

% external ip address - one attatched with your house
% internal ip address - one attatched with the devices using the extermal ip in the house * internet never sees this, only used by our router

% your home ip address is not fixed and can change overnight (big companies, however, often have static IP addresses)

% https://www.youtube.com/watch?time_continue=28&v=sDmErvLJSnc

\begin{frame}[fragile]\ft{Internet terminology basics}
\vfill
A \define{uniform resource locator} (URL) is an address of an item on the Internet. A URL has three parts:
\begin{enumerate}
\vfill
\item Protocol: \url{http://} Hypertext Transfer Protocol: Tells the computer how to handle the file
\item Server computer's domain name or IP address
\item Item's path and name:
Tells the server which item (file, page, resource) is requested and where to find it.
\end{enumerate}
\vfill
Example:
\begin{Verbatim}[xleftmargin=.5in,commandchars=\\\{\}]
\textcolor{brown}{http://}\green{people.ok.ubc.ca}\purple{/ivrbik/teaching.html}
\end{Verbatim}
 \textcolor{brown}{http protocol} \green{server domain name} \purple{location of file/resource on server}
 \vfill
\end{frame}






\begin{frame}[fragile]\ft{Accessing ({\tt GET}) Web Sites via URL with Python}
{\bf urllib} is a package that collects several modules for working with URLs: 
\begin{description}
\item[urllib.request] for opening and reading URLs, more \href{https://docs.python.org/3/library/urllib.request.html#module-urllib.request}{here}
\item[urllib.parse] for parsing URLs; more  \href{https://docs.python.org/3/library/urllib.parse.html#module-urllib.parse}{here}
\end{description}

\begin{Verbatim}[xleftmargin=.1in,commandchars=\\\{\}]
\purple{import urllib.request}
# dont forget the http://
loc="http://google.com"
\red{site = urllib.request.urlopen(loc)}
contents = site.read()
print(contents)
\red{site.close()}
\end{Verbatim}
\end{frame}
  

\begin{frame}[fragile]\ft{Engine Search with Python }
%{\tt Request} objects also allow us to:
%\begin{itemize}
%\item pass data to be sent to the server
%\item pass ``header'' containing metadata 
%\end{itemize}

The URL (that we have input as a string) must be properly URL encoded.\nl 
Rather than learning the URL encoding language, we can use \command{urllib.parse.urlencode} to do it for us.\nl% encode to url language\nl %(eg. code spaces as \%20)

{\bf Notes:}
\begin{itemize}
\item \href{https://requests.readthedocs.org/en/master/}{Requests} package is even higher-level and easy to use\nl
\item \command{urllib.request.urlopen} could also take a {\tt Request} object (like a .txt document which can be {\tt read}) as an argument.\nl
\end{itemize}
\end{frame}




\begin{frame}[fragile]
To get a shallow understanding of what is going on under the hood, go to \url{www.ask.com} and search (or query) \verb|"data analysis"|.\nl
You'll get a new URL that might look something like this:\nl
\begin{allintypewriter}
\quad https://www.ask.com/web?\red q=data+analysis\nl
\end{allintypewriter}

In the above the \red{\tt q} is a parameter, with the input \verb|data+analysis|\nl

We will save this information in  a python dictionary (called {\tt values} on the next slide) and pass it to the urlencoder:
\begin{FVerbatim}
>>> values = {'q':'data analysis'}
>>> data = urllib.parse.urlencode(values)
>>> print(data)
q=data+analysis
>>> data = data.encode('utf-8') 
>>> print(data)
b'q=data+analysis'
\end{FVerbatim}


\end{frame}



\begin{frame}[fragile]\ft{Engine Search with Python ({\sf POST} request)}
 %commandchars=\\\{\}]
\begin{Verbatim}[fontsize=\small, commandchars=\\\{\} ]
import urllib.parse
import urllib.request

url = 'https://www.ask.com/web'
# Build and encode data
values = \{'q':'data analysis'\} 

data = urllib.parse.urlencode(values)
data = data.encode('utf-8') 
req = urllib.request.Request(url, data)
with urllib.request.urlopen(req) as response:
    page = response.read()
    print(page)
\end{Verbatim}

%\begin{Verbatim}[fontsize=\small, commandchars=\\\{\} ]
%import urllib
%url = "http://www.google.com/search?hl=en&q=data+analysis" 
%headers=\{'User-Agent':'Mozilla/5.0 (Windows NT 6.1)'\} 
%\purple{request = urllib.request.Request(url,None,headers)}
%response = urllib.request.urlopen(request)
%data = response.read()
%data = data.decode() # Convert from Unicode to ASCII print(data)
%response.close()
%\end{Verbatim}

\end{frame}


\begin{frame}
\begin{itemize}
\item Right now we are just looking at the source and the content is in a form that is not very useful to us.\nl
\item Our browser typically does the heavy lifting in order to convert that mess to something readable.\nl
\item In the next lecture we'll see some useful tools for parsing though this data to extract the useful information from websites.
\end{itemize}
{\bf Notes:}
\begin{itemize}
\item \href{https://www.pythonforbeginners.com/beautifulsoup/beautifulsoup-4-python}{BeautifulSoup} library to make easier to extract information from HTML pages.\end{itemize}

\end{frame}




%\begin{frame}[fragile]\ft{Sending data ({\tt PUT}) to URL with Python}
%%commandchars=\\\{\}]
%
%
%\begin{Verbatim}[fontsize=\small, commandchars=\\\{\} ]
%import urllib.parse
%import urllib.request
%
%url = 'http://data301.ok.ubc.ca/ivrbik/tomcat/provinceState.jsp'
%headers=\{'User-Agent':'Mozilla/5.0 (Windows NT 6.1)'\} 
%# Build and encode data
%values = \{'country' : 'US'\}
%data = urllib.parse.urlencode(values)
%data = data.encode('ascii')
%req = urllib.request.Request(url, data, headers)
%with urllib.request.urlopen(req) as response:
%   page = response.read()
%   print(page)
%\end{Verbatim}
%Note: {\bf BeautifulSoup} library to make easier to extract information from HTML pages.
%\end{frame}





%%%%%%%%%%%%%%%%%%%%%%%%%

% question:

\begin{frame}[fragile]\ft{}
  \begin{example}
How many of the following statements are TRUE?
\begin{enumerate}
\item {{An IPv4 address has 4 numbers between 0 and 256 inclusive.}}
\item {{A domain name is hierarchical with most specific part at the end.}}
\item {{Typically, a URL will reference more than one resource/item.}}
\item {{Python uses the file module for accessing URLs.}} 
\end{enumerate}
\begin{multicols}{5}
\begin{enumerate}[A)]
\item 0 
\item 1
\item 2
\item 3
\item 4
\end{enumerate}
\end{multicols}
  \end{example} 
\end{frame}


% answer:

\begin{frame}<handout:0>[fragile]\ft{}
  \begin{block}{Answer:}
How many of the following statements are TRUE?
\begin{enumerate}
\item {\color<1->{red}	{An IPv4 address has 4 numbers between 0 and 25}\green{5} {inclusive.}}
\item {\color<2->{red}	{A domain name is hierarchical with most specific part at the end.}}
\item {\color<3->{red}	{Typically, a URL will reference more than one resource/item.}}
\item {\color<4->{red}	{Python uses the file module for accessing URLs.}} 
\end{enumerate}
\begin{multicols}{5}
\begin{enumerate}[A)]
\item \tans{4}{0} 
\item 1
\item 2
\item 3
\item 4
\end{enumerate}
\end{multicols}
  \end{block} 
\end{frame}


\begin{frame}\ft{Try it}
\begin{example}
\begin{enumerate}
\item  Write a Python program that connects to any web page and prints its contents.
\item Write a Python program that connects to: 
\url{http://www.sharecsv.com/dl/ab69f200ce5071b27e4af626da293d27/province_population.csv}
and outputs the CSV data.
Modify your program to print each province and its 2015 population in descending sorted order. (See next slide for hint)
\end{enumerate}
\end{example}
\end{frame}

\begin{frame}{Advanced sorting}
\begin{itemize}
\item Each element in {\tt provinces} list is yet another list containing the province's population and name
\vfill
\item  Calling {\tt sort} on this list will sort according to that \textit{first} element (i.e. population)
\vfill
\item For advanced sorting, we could also specify a function as a second argument in {\tt sort}.  General syntax:
\begin{center}
\textit{<list>}{\tt .sort(reverse=\textit{<True/False>}, key= myFunc)}
\end{center}
\vfill
\item For more on this topic see \href{https://www.w3schools.com/python/ref\_list\_sort.asp}{w3schools} and \href{https://www.geeksforgeeks.org/python-list-sort/}{GeeksforGeeks}
\end{itemize}

\end{frame}




\end{document}

